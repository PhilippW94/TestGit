\documentclass[12pt]{article}

\usepackage{amsmath}
\usepackage{graphicx}
\usepackage[utf8]{inputenc}
\usepackage{url}
\usepackage{mathtools}

\usepackage{braket}
\usepackage{enumerate}
\usepackage{amssymb}
\usepackage{float}
\usepackage{geometry} 
\geometry{a4paper, top=30mm, left=20mm, right=20mm, bottom=30mm}
\usepackage{afterpage}


\usepackage{fancyhdr}
\setlength{\headheight}{23.1pt}

\textwidth460pt
\oddsidemargin5pt

 
\pagestyle{fancy}
\fancyhf{}
\rhead{Philipp Waffler}
\cfoot{\thepage}

\newcommand*\Laplace{\mathop{}\!\mathbin\bigtriangleup}
\newcommand*\DAlambert{\mathop{}\!\mathbin\Box}
\begin{document}

\section{Statistische Begründung der Thermodynamik }

Statistische Interpretation der statistischen Mechanik grundsätzlich anders als die der Quantenmechanik!

Thermodynamisches Gleichgewicht wird von system makroskopischer Größe angestrebt. Dieses Gleichgewicht ist unabhängig von den Anfangsbedingungen des Systems.


Statistische Interpretation der statistischen Mechanik grundsätzlich anders als die der Quantenmechanik!

Thermodynamisches Gleichgewicht wird von system makroskopischer Größe angestrebt. Dieses Gleichgewicht ist unabhängig von den Anfangsbedingungen des Systems.


\end{document}



























